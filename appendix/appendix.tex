% !TEX encoding = UTF-8
\addcontentsline{toc}{chapter}{附录A~~攻读硕士学位期间发表论文及科研工作情况}%添加到目录中
\chapter*{附录A~~~~攻读硕士学位期间发表论文及科研工作情况}
\appedixfigtabnum{A}%重新计算附图和表的标题号和计数号,参数是附录A,B或者C...
\setlength{\parindent}{0em}
(一)发表的学术论文
\begin{publist}
	\item XXX,XXX. Density and Non-Grid based Subspace Clustering via Kernel Density Estimation[C]. ECML-PKDD 2012, Bristol, UK.(Submitted, Under review)
	\item XXX,XXX. A tree parent storage based on hashtable for XML construction[C]. Communication Systems, Networks and Applications, Hongkong, 2010: 325-328. (EI DOI: 10.1109/ICCSNA.2010.5588732)
\end{publist}
\vspace*{1em}
\textbf{(二)申请及已获得的专利(无专利时此项不必列出)}
\begin{publist}
	\item XXX,XXX. XXXXXXXXX:中国,1234567.8[P]. 2012-04-25.
\end{publist}
\vspace*{1em}
\textbf{(三)参与的科研项目}
\begin{publist}
	\item XXX,XXX. XX~信息管理与信息系统, ~国家自然科学基金项目.课题编号:XXXX.
\end{publist}
%\vfill
\hangafter=1\hangindent=2em\noindent
\setlength{\parindent}{2em}

附录是正文主体的补充。下列内容作为附录。
\begin{itemize}
	\item 攻读学位期间发表的(含已录用,并有录用通知书的)与学位论文相关的学术论文。必须另页单列论文目录,格式同“参考文献”。
	\item 由于篇幅过大,或取材于复制件不便编入正文的材料、数据。
	\item 对本专业同行有参考价值,但对一般读者不必阅读的材料。
	\item 论文中使用的符号意义、单位缩写、程序全文及有关说明书。
	\item 附件:计算机程序清单、软磁盘、鉴定证书、获奖奖状或专利证书的复印件等。
\end{itemize}

对于一些不宜放入正文中、但作为毕业论文(设计)又是不可缺少的部分,或有重要参考价值的内容,可编入毕业论文(设计)的附录中。例如,过长的公式推导、重复性的数据、图表、程序全文及其说明等。论文的附录依序用大写正体A,B,C……编序号,如:附录A。附录中的图、表、式等另行编序号,与正文分开,也一律用阿拉伯数字编码,但在数码前冠以附录序码,如:图A1;表B2;式(B3)等,

这个示例为插入图片:
\begin{figure}[H]
	\centering
	\includegraphics[width=0.618\textwidth]{figure.jpg}%图片名称,放在/figures目录下
	\caption{图片插入\label{fig:appA}}
\end{figure}

\begin{table}[H]
	\begin{center}
		\caption{希腊字母表\label{tab:appA}}
		\begin{tabular}{|c|c|c|c|c|}
			\hline
			Alpha    & Beta    & Gamma    & Delta    & Theta    \\
			\hline
			$\alpha$ & $\beta$ & $\gamma$ & $\delta$ & $\theta$ \\
			\hline
			$A$      & $B$     & $\Gamma$ & $\Delta$ & $\Theta$ \\
			\hline
		\end{tabular}
	\end{center}
\end{table}

\begin{equation}
	\hat{H}=\frac{\epsilon}{2}\hat{\sigma}_{z}-\frac{\Delta}{2}\hat{\sigma}_{x}+\sum_{k}\omega_{k}\hat{b}_{k}^{\dagger}\hat{b}_{k}+\sum_{k}\frac{g_{k}}{2}\hat{\sigma}_{z}(\hat{b}_{k}+\hat{b}_{k}^{\dagger})\label{eq:appA}
\end{equation}

\addcontentsline{toc}{chapter}{附录B~~本论文相关源代码}%添加到目录中
\chapter*{附录B~~~~本论文相关源代码}
\appedixfigtabnum{B}%重新计算附图和表的标题号和计数号
这个示例为插入图片:
\begin{figure}[H]
	\centering
	\includegraphics[width=0.618\textwidth]{figure.jpg}%图片名称,放在/figures目录下
	\caption{图片插入\label{fig:appB}}
\end{figure}

\begin{table}[H]
	\begin{center}
		\caption{希腊字母表\label{tab:appB}}
		\begin{tabular}{|c|c|c|c|c|}
			\hline
			Alpha    & Beta    & Gamma    & Delta    & Theta    \\
			\hline
			$\alpha$ & $\beta$ & $\gamma$ & $\delta$ & $\theta$ \\
			\hline
			$A$      & $B$     & $\Gamma$ & $\Delta$ & $\Theta$ \\
			\hline
		\end{tabular}
	\end{center}
\end{table}

\begin{equation}
	\hat{H}=\frac{\epsilon}{2}\hat{\sigma}_{z}-\frac{\Delta}{2}\hat{\sigma}_{x}+\sum_{k}\omega_{k}\hat{b}_{k}^{\dagger}\hat{b}_{k}+\sum_{k}\frac{g_{k}}{2}\hat{\sigma}_{z}(\hat{b}_{k}+\hat{b}_{k}^{\dagger})\label{eq:appB}
\end{equation}

原创性声明和版权使用授权书:按统一格式(见附件6)书写打印,并由论文作者本人签名。