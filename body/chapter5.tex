% !TEX encoding = UTF-8
\chapter{简要版帮助}
若您已很熟悉\LaTeX~的相关操作,建议阅读这份简要帮助说明。
\section{模板文件结构\label{sec:files}}
整个模板根目录的文件列表如下:
\begin{center}
	\begin{tabular}{|l|p{7.5cm}|l|}
		\hline
		HUNNUthesis.tex          &\TeX{}样例文件             & \textcolor{red}{{*}} \\
		\hline
		HUNNUthesis.cls&包含论文所使用的宏包和全文格式的定义。& \textcolor{red}{{*}} \\
		\hline
		HUNNUThesis.tex& 主文件,包含封面、扉页和其他章节的引用信息。& \textcolor{red}{{*}} \\
		\hline
		preface& 包含毕业设计论文的中英文摘要。& \textcolor{red}{{*}} \\
		\hline
		images& 包含封面用到的湖南师范大学Logo。& \textcolor{red}{{*}} \\
		\hline
		figures& 包含正文中所用到的图片。& \textcolor{red}{{*}} \\
		\hline
		body& 包含正文的所有章节。& \textcolor{red}{{*}} \\
		\hline
		appendix& 附件相关内容& \textcolor{red}{{*}} \\
		\hline
		appendix.tex & 作者的发表论文和参加科研情况说明& \textcolor{red}{{*}} \\
		\hline
		acknowledgements.tex&致谢文件& \textcolor{red}{{*}} \\
		\hline
		statement.tex&原创性说明和版权使用授权说明书。& \textcolor{red}{{*}} \\
		\hline
		hunnubib.bst             & 参考文献样式文件               & \textcolor{red}{{*}} \\
		\hline
		references/reference.bib & bib数据库                  & \textcolor{red}{{*}} \\
		\hline
		official\_documents& 学位论文的撰写格式和开题报告实施管理办法。& \\
		\hline
		clean.bat& 双击此文件,可以用来清理HUNNUThesis.tex在编译之后生成的所有附属文件,如后缀名为.aux,.log,.bak的文件。&  \\
		\hline
	\end{tabular}
\end{center}
注: \textcolor{red}{{*}} 表示\LaTeX{}模板必须的文件。
\subsection{文献引用}
将引文的bib数据库(默认文件名为reference.bib)放入模板根目录下的references文件夹,即可通过插入---文献引用自动产生引文。
\begin{itemize}
	\item 参考文献上标引用
	\begin{itemize}
		\item Journal:An article \upcite{ELIDRISSI94,MELLINGER96}。%参考文献上标
		\item An book \upcite{IEEE-1363,tex,companion}。%参考文献正常引用
		\item Conference:A conference \upcite{kocher99,DPMG,cnproceed}。
		\item Manual:A manual\upcite{NPB2}。
		\item MasterThesis:\upcite{zhubajie,metamori2004,shaheshang,FistSystem01}。
	\end{itemize}
	\item 参考文献正常引用
	\begin{itemize}
	\item Journal:An article \cite{ELIDRISSI94,MELLINGER96}。%参考文献上标
	\item An book \cite{IEEE-1363,tex,companion}。%参考文献正常引用
	\item Conference:A conference \cite{kocher99,DPMG,cnproceed}。
	\item Manual:A manual\cite{NPB2}。
	\item MasterThesis:\cite{zhubajie,metamori2004,shaheshang,FistSystem01}。
\end{itemize}
\end{itemize}
\subsection{伪代码实现}
\begin{algorithm}
	\caption{放进冰箱的大象}\label{算法实例}
	\begin{algorithmic}
		\REQUIRE 有一只大象
		\ENSURE 放进冰箱里
		\FOR {没有剩余的大象}
		\IF {大象比冰箱大}
		\STATE 把大象分割
		\ENDIF
		\ENDFOR
		\STATE 第一步
		\STATE 第二步
		\STATE 第三步
	\end{algorithmic}
	AAA\end{algorithm}
\subsection{代码展示}
可以把你的程序添加到附录里,展示自己的工作。
\begin{lstlisting}[language={[ANSI]C},
numbers=left,
numberstyle=\tiny,
basicstyle=\small\ttfamily,
stringstyle=\color{purple},
keywordstyle=\color{blue}\bfseries,
commentstyle=\color{olive},
directivestyle=\color{blue},
showstringspaces=false]
	#include <stdio.h>
	int main(int argc, char ** argv)
	{
		/*打印Hello,world*/
		printf("Hello, world!\n");
		
		return 0;
	}
\end{lstlisting}
\section{依赖}
HUNNUthesis依赖于以下宏包,这些宏包在常见的\LaTeX{}发行版中都包括,在安装使用之前,请确定你的\TeX{}发行版中都已正常安装这些宏包
\begin{table}[H]
	\centering
	\begin{tabular}{cccc}
\hline
{natbib} & {amsmath} & {amsfonts} & {amssymb} \\

{graphicx} & {subfigure} & {mathptmx} & {float} \\

{fontenc} & {booktabs} & {setspace} & {listings} \\

{xcolor} & {multirow} & {fancyhdr} & {etoolbox} \\

{tocloft} & {array} & {makecell} & {forloop} \\

{xstring} & {hyperref} & {cleveref} & {enumitem} \\

{algorithm} & {algorithmic} & {caption} & {ifthen} \\

{titlesec} & {ulem} & {amssymb} & {wasysym} \\

{flafter} & {booktabs} & {longtable} & {tabularx} \\

{setspace} & {subfigure} & {enumitem} & {calc } \\

{txfonts} & {bm} & {ntheorem} & {fancyvrb} \\

{xcolor} & {--} & {--} & {--} \\
\hline
	\end{tabular}
\end{table}
如果你尚未安装这些宏包,可以启动你的 \TeX{} 发行版的宏包管理器
来安装;或者到 \url{http://www.ctan.org} 上搜索下载并安装。
\section{基本设置}
\begin{enumerate}
	\item 论文正文所用图片搜索路径默认设置为模板根目录下的figures/。
	\item bib数据库默认设置为模板根目录下的references/reference.bib。 其中bib文件可由任意文献库管理软件自动生成
\end{enumerate}
\section{文字命令}
\subsection{常用命令}
\LaTeX 提供了一系列命令,用于修改字体、字号、数字等的呈现形式。
\subsubsection{字体}
ctex宏包及其文档类可直接使用以下六种字体:
\vspace{1em}\noindent\hrule
\begin{verbatim}
宋体: \songti       启用宋体  {\songti 宋体}。
黑体: \heiti        启用黑体  {\heiti 黑体}。
仿宋: \fangsong    启用仿宋   {\fangsong 仿宋}。
楷书: \kaishu       启用楷书   {\kaishu 楷书}。
隶书: \lishu        启用隶书  {\lishu 隶书}。
幼圆: \youyuan     启用幼圆   {\youyuan 幼圆}。
\end{verbatim}
\noindent\hrule\vspace{1em}

宋体:{\songti 宋体};黑体:{\heiti 黑体};仿宋:{\fangsong 仿宋};楷书:{\kaishu 楷书};隶书: {\lishu 隶书};幼圆: {\youyuan 幼圆}。
\subsubsection{字号}
{\bf 字号}%

\begin{center}
	\begin{tabular}{cccccccc}
		\toprule
		初号 & 小初 & 一号 & 小一 & 二号 & 小二 & 三号 & 小三 \\
		0    & -0   & 1    & -1   & 2    & -2   & 3    & -3   \\
		\hline
		四号 & 小四 & 五号 & 小五 & 六号 & 小六 & 七号 & 八号 \\
		4    & -4   & 5    & -5   & 6    & -6   & 7    & 8    \\
		\bottomrule
	\end{tabular}
\end{center}

\vspace{1em}\noindent\hrule
\begin{verbatim}
{\zihao{0}初号}; \dots {\zihao{4}四号};\dots 
\zihao{7}{七号} \zihao{4}
\end{verbatim}
\noindent\hrule\vspace{1em}

{\zihao{0}初号}; \dots {\zihao{4}四号};\dots \zihao{7}{七号}
\zihao{4}