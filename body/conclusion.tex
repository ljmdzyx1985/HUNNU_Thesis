% !TEX encoding = UTF-8
\addcontentsline{toc}{chapter}{结论}
\chapter*{结\quad 论}
结论单独作为一章排写,但不加章号。结论是毕业论文(设计)的总结,是整篇论文(设计)的归宿。要求精炼、准确地概述全文的主要观点:或自己赞成的观点、或自己反对的观点、或自己的创造性工作与新的见解及其意义和作用,还进一步提出需要讨论的问题和建议等。特别提醒这部分不是写你做毕业设计(论文)的感想,不是抒情。下面是一个分页符,可以另起一页

结论应是作者在学位论文研究过程中所取得的创新性成果的概要总结,不能与摘要混为一谈。
学位论文结论应包括论文的主要结果、创新点、展望三部分,在结论中应概括论文的核心观点,
明确、客观地指出本研究内容的创新性成果(含新见解、新观点、方法创新、技术创新、理论创新),并指出今后进一步在本研究方向进行研究工作的展望与设想。
对所取得的创新性成果应注意从定性和定量两方面给出科学、准确的评价,分(1)、(2)、(3)…条列出,宜用“提出了”、“建立了”等词叙述。

结论应是作者在学位论文研究过程中所取得的创新性成果的概要总结,不能与摘要混为一谈。
学位论文结论应包括论文的主要结果、创新点、展望三部分,在结论中应概括论文的核心观点,
明确、客观地指出本研究内容的创新性成果(含新见解、新观点、方法创新、技术创新、理论创新),并指出今后进一步在本研究方向进行研究工作的展望与设想。
对所取得的创新性成果应注意从定性和定量两方面给出科学、准确的评价,分(1)、(2)、(3)…条列出,宜用“提出了”、“建立了”等词叙述。

结论应是作者在学位论文研究过程中所取得的创新性成果的概要总结,不能与摘要混为一谈。
学位论文结论应包括论文的主要结果、创新点、展望三部分,在结论中应概括论文的核心观点,
明确、客观地指出本研究内容的创新性成果(含新见解、新观点、方法创新、技术创新、理论创新),并指出今后进一步在本研究方向进行研究工作的展望与设想。
对所取得的创新性成果应注意从定性和定量两方面给出科学、准确的评价,分(1)、(2)、(3)…条列出,宜用“提出了”、“建立了”等词叙述。

结论应是作者在学位论文研究过程中所取得的创新性成果的概要总结,不能与摘要混为一谈。
学位论文结论应包括论文的主要结果、创新点、展望三部分,在结论中应概括论文的核心观点,
明确、客观地指出本研究内容的创新性成果(含新见解、新观点、方法创新、技术创新、理论创新),并指出今后进一步在本研究方向进行研究工作的展望与设想。
对所取得的创新性成果应注意从定性和定量两方面给出科学、准确的评价,分(1)、(2)、(3)…条列出,宜用“提出了”、“建立了”等词叙述。